%------------------------------------------------
% Abstract Template 
% Sam Dougherty
% Last edited 12.2.2015
%------------------------------------------------
\documentclass[10pt,letterpaper]{article}
%------------------------------------------------
% 		Packages sam
%------------------------------------------------
\usepackage{geometry}  					% For paper options
\usepackage{graphicx}					% For graphics
\usepackage{setspace}					% For setting spacing \doublespaceing, \onehalfspacing, \singlespacing, etc
\usepackage{fullpage}					% Makes margins equal
%------------------------------------------------
%	Setup of things
%-----------------------------------------------
\geometry{letterpaper}    				% Paper size
%\geometry{landscape}                	% Activate for for rotated page geometry
\parskip=0pt							% Sets space between paragraphs
\setlength{\textfloatsep}{					
10pt plus 1.0pt minus 2.0pt}				

%------------------------------------------------------------------------------------------------------------------------------------------------
%	This is the beginning
%------------------------------------------------------------------------------------------------------------------------------------------------
\begin{document}
\begin{flushleft}
% Title 
\textbf{Whitty Title: With a colon}
\newline
\small{\textbf{sub title if necessary, but if you do your messing something up}}
\vspace{1em}

% Authors
\textbf{Last, First; Last, First; Last,First; etc.}
\rule{\textwidth}{1pt}
\vspace{2em}

% Abstract
\onehalfspacing % manipulated line spacing

Understanding advection and removal of nitrate in streams is relevant to managing anthropogenic changes to the global nitrogen cycle. However, difficulty measuring stream denitrification limits the extent of direct measurements.  Natural relationships between denitrification and more easily measured quantities may allow for extensive estimation of denitrification rates. Using results from the Second Lotic Intersite Nitrogen experiment (LINX-II), we examined relationships among denitrification rate, nitrate uptake rate, metabolic rates and stream concentrations of nitrate and DOC. Stream nitrate concentrations and respiration rates explain much of the variance in observed denitrification rates of 49 streams $(r2 = 0.62, p<0.01)$ distributed across the continental U.S. and Puerto Rico. Based on this relationship, we propose a novel simulation model structure that links carbon and nitrogen dynamics in stream networks. 
\vspace{2em}
\rule{\textwidth}{1pt}
\today
\end{flushleft}
\end{document}












